\chapter{Prehľad algoritmov}
Na hľadanie najkratších ciest v grafe poznáme mnoho algoritmov, ktoré vieme rozdeliť do troch skupín.


\begin{itemize}
\item Point To Point Shortest Path - hľadáme najkratšiu cestu medzi dvoma zadanými bodmi
\item Single Source Shortest Path - pre daný vrchol {\sl v} hľadáme najkratšiu cestu do všetkých vrcholov grafu.
\item All Pairs Shortest Path - skúmame najkratšiu cestu medzi všetkými dvojicami vrcholov.
\end{itemize}

Napriek tomu, že sú tieto problémy na obecných grafoch NP-ťažké, na herných mapách, kde majú všetky vrcholy kladnú cenu, vieme nájsť riešenie v polynomiálnom čase.
V práci sa ďalej budeme zaoberať riešením prvého problému (Point to Point Shortest Path).

V tejto kapitole si následne popíšeme algoritmy, ktoré sú použiteľné na všetkých grafoch 
s nezápornými dĺžkami hrán.

\section{Dijkstrov algoritmus}
Medzi základné algoritmy patrí Dijkstrov algoritmus, ktorý je asymptoticky optimálny (TODO?? for sure?).

Pri hľadaní cesty z vrcholu {\sl s} do vrcholu {\sl t} prechádzame postupne vrcholy zo stúpajúcou vzdialenosťou od {\sl s}, až dokým sa nedostaneme k cieľovému vrcholu {\sl t}.
Graficky si beh algoritmu môžme predstaviť ako kruh so stredom v bode {\sl s} so zväčšujúcim sa polomerom. Algoritmus napísaný v pseudokóde je nasledovný:

\begin{lstlisting}
d(s) <- 0
d(*) <- $\infty$


\end{lstlisting}



