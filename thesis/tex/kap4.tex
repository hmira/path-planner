\chapter{Testovanie a výsledky}

\section{Kritériá a popis testovania}
\subsection{Vstupné dáta}
Častým problémom pri vzájomnom porovnávaní algoritmov je
nájsť testovaciu vzorku, ktorá mi otestovala beh algoritmu na širokej škále grafov.
Ako bolo v úvode spomenuté, súťaž {\sl Grid-Based Path Planning Competition} poskytuje množstvo máp rôznych typov a rozmerov,
na ktorých sa algoritmy dajú testovať. Konkrétny popis typov máp sa nachádza v \cite{sturtevant2012benchmarks}.

\subsection{Testovacie kritériá}
Testovacie kritéria budú podobné ako testovacie kritériá v súťaži. Budú sa však mierne líšiť, pretože algoritmus nebol 
naplánovaný tak, aby .... TODO?? (prvych 20 kroko a nezaujima, na druhu stranu ma zaujima pocet prehladanych vrcholov...)

\begin{itemize}
\item Počet prehľadaných vrcholov.
\item Rýchlosť nájdenia cesty.
\end{itemize}


\subsection{Testované algoritmy}
Testovať budeme nasledujúce algoritmy:
\begin{itemize}
\item Dijkstrov algoritmus nad binárnou haldou.
\item Dijkstrov algoritmus nad priehradkovou haldou.
\item A* používajúc rôzny počet landmarkov.
\item A* používajú optimálny počet landmarkov a obdĺžnikovú dekompozíciu.
\item TODO?? nieco dalsie blablabla - cudzie algoritmy.
\end{itemize}


\subsection{Kompilácia}
Kód bude kompilovaný kompilátorom g++. Bude porovnaná
rýchlosť behu programu pri kompilácii s týmito direktívami.
\begin{itemize}
\item -O1
\item -O2
\item -O3
\item -O3 -march=native
\end{itemize}
ASK?? toto bude len pri jednom algoritme pri jednom grafe, nie? neni to zbytocne???

\subsection{Typy map a ciest}
TODO?? - kratke dlhe, klukate a rovne obrazok asopn 3 map





(ASK?? poet prehladanych vrcholov nie je kriterium, ktore dokazem testovat na cudzich algoritmoch - musel by smo do nich zahat a pridavat tam funkciu, ktora to pocita - co mam robit? mam pocet vrcholov testovat len na mojich algoritmoch?)




Na porovnávanie využijeme benchmark

TODO?? bibliografia styl - priezvisko,meno - alebo naopak???