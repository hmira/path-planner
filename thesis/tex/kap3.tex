\chapter{Nový algoritmus: NovellA*}

\section{Zlepšenie výkonu v niektorých prípadoch}
Nie všetky cesty sú také kľukaté. V niektorých prípadoch
počiatočný a koncový bod ležia blízko pri sebe a nemá zmysel a cesta sa nájde 
veľmi jednoducho - ako "kvazieuklidovská" vzdialenost.

??TODO: prepisat
Pre tento prípad sa pokúsime rozložiť mapu na obdĺžniky
(prípadne konvexné obdĺžniky, kde nenasledujú za sebou dve \dots).

V tomto pripade nam to zaberie cca 5 mb na ulozenie mapy,...


Vyššie popísaný algoritmus sa nám oplatí použivať v obecných prípadoch.
V špeciálnych prípadoch totiž vieme použiť niečo efektívnejšie.

Predstavme si, že máme obdĺžnikovú mapu bez nepriechodných oblastí. V takomto prípade vôbec nepotrebujeme
prechádzať graf 
celý, ale intuitívne vidíme, že nám stačí ísť šikmo a potom rovno.

Túto vec použ
\subsection{Navigácia}

