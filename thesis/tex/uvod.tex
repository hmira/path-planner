\chapter*{Úvod}
\addcontentsline{toc}{chapter}{Úvod}

\section{Motivácia}
Hľadanie najkratšej cesty je jedným z elementárnych problémov teórie grafov.
Dôležitosť tohto problému je zrejmá, najmä keď si uvedomíme jeho všestranné využitie, napríklad v 
umelej inteligencii, v počítačovych hrách a podobne. Nanešťastie riešenia a algoritmy využívané v komerčných hrách 
sú closed-source a teda obecne nie známe.

\section{Stručný popis práce, ciele}
Na nájdenie najkratšej cesty v obecných grafoch využívame Dijkstrov algoritmus, prípadne jeho nadstavby, ktoré si následne rozoberieme.

Cieľom práce je hľadanie cesty v špeciálnych grafoch - tzv. herných mapách. 
Zadanie tohto problému sa od všeobecneho hľadania najkratšej cesty v obecných grafov líši v dvoch veciach.
Jednak grafy herných máp predstavujú akúsi mriežku - a teda arita vrcholov je maximálne 8 a jednak pri riešení tejto úlohy máme k dispozícii čas
na predspracovanie mapy a vytvorenie pomocných dátových štruktúr.


V práci sme naimplementovali vlastný algoritmus a porovnali ho s doterajšímy známymi.
TODO?? Počas práce sme prišli na zaujímavé zefektívnenie algoritmov [snad na nieco prijdem :))]a dúfame v jeho rozšírenie do hernej sféry.

ASK?? ake su vlastne ciele? mam vymysliet vzbrusu novy algoritmus?

V prvej kapitole si zadefinujeme kľúčové termíny a popíšeme problem. Na konci kapitoly spomenieme súťaž, ktorej sa daný algoritmus zúčastnil 
a popíšeme jej podmienky.
Druhá kapitola sa pokúsime rozobrať doterajšie zistenia a algoritmy používané na riešenie obdobných problémov.
V tretej kapitole popíšeme náš algoritmus a vo štvrtej kapitole ho porovnáme s ostatnými algoritmami a uvedieme výsledky.
