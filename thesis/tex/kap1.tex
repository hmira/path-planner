\chapter{Zadanie problému a cieľové požiadavky}

\section{Úvodné definície a značenia}
Na začiatok si zaveďme niektoré dôležité pojmy teórie grafov.
Budú sa týkať obecnej teórie a úlohu so všetkými jej špecifikami si ozrejmíme v nasledujúcej kapitole.
\begin{define}
{\sl Graf G} je usporiadaná dvojica (V, E), kde V označuje množinu vrcholov(vertices) a $E \subseteq V \times V $ označuje množinu hrán (edges). Značíme G = (V, E).
\end{define}

\begin{define}
{\sl Ohodnotený graf (G, w)} je graf s spolu s reálnou funkciou
$w: E(G) \to \R$, kde $w$ je funkcia, ktorá každej hrane priradí
reálne číslo, takzvanú \emph{cenu}, alebo \emph{váhu} hrany.
\end{define}


Teraz keď už vieme, čo je to graf, skúsme si zadefinovať najkratšiu cestu.

\begin{define}
{\sl Ohodnotený graf (G, f)} je graf s spolu s reálnou funkciou
$f: E(G) \to \R$, kde $f$ je funkcia, ktorá každej hrane priradí
reálne číslo, takzvanú \emph{cenu}, alebo \emph{váhu} hrany.
\end{define}



\section{Název druhé podkapitoly v první kapitole}

