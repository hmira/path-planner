\chapter{Zadanie problému a cieľové požiadavky}

\section{Úvodné definície a značenia}
Na začiatok si zaveďme niektoré dôležité pojmy teórie grafov.
Budú sa týkať obecnej teórie a úlohu so všetkými jej špecifikami si ozrejmíme v nasledujúcej kapitole.
\begin{define}
{\sl Graf G} je usporiadaná dvojica (V, E), kde V označuje množinu vrcholov(vertices) a $E \subseteq V \times V $ označuje množinu hrán (edges). Značíme G = (V, E).
\end{define}

\begin{define}
{\sl Ohodnotený graf (G, w)} je graf s spolu s reálnou funkciou (tzv. ohodnotením)
$w: E(G) \to \R$, kde $w$ je funkcia, ktorá každej hrane priradí
reálne číslo, takzvanú \emph{cenu}, alebo \emph{váhu} hrany.
\end{define}


Teraz keď už vieme, čo je to graf, skúsme si zadefinovať najkratšiu cestu. Začnime najprv obecne cestou.

\begin{define}
{\sl Cesta P z vrcholu $v_0$ do vrcholu $v_n$ v grafe G } je postupnosť $P = (v_{0},e_{1},v_{1},\dots, e_{n}, v_{n})$,
pre ktorú platí $e_{i} = \{v_{i-i},v_{i}\}$ a taktiež
$v_{i} \ne v_{j}$ pre každé $i \ne j$.
\end{define}

Všimnime si, že v ceste nenavštívime žiaden vrchol dvakrát a teda cesta neobsahuje kružnice.

\begin{define}
{\sl Cena cesty P z vrcholu $v_0$ do vrcholu $v_n$ v ohodnotenom grafe (G, w) } je súčet cien hrán, ktoré sa na ceste nachádzajú.
\end{define}

\begin{define}
{\sl Najkratšia cesta P z vrcholu $v_0$ do vrcholu $v_n$
v ohodnotenom grafe (G, w)} 
je cesta s najnižsou cenou.
\end{define}





\section{Herná mapa}

Po zavedení kľúčových pojmov sa dostávame k samotnému zadaniu úlohy. 
Ako sme už spomínali, problém budeme riešit na tzv. herných mapách. V čom sa herná mapa od obecného grafu odlišuje?

Ide v podstate o graf dosť monotónny a obmedzený. Vizuálne si ju môžme predstaviť ako graf v ktorom sú vrcholy rozostúpené v tvare mriežky a hrana
je stále medzi dvojicami susedných vrcholov vo všetkých ôsmych smeroch. Cena vodorovnej alebo zvislej hrany je $1$ a cena šikmej hrany je $\sqrt{2}$.
V práci budeme ale miesto iracionálneho čísla $\sqrt{2}$ použivať konštantu 1.4142, nakoľko táto konštatna sa používa aj v samotnej súťaži.
FIXME??: Rozdiel medzi takto zavedenou konštantou a skutočnou hodnotou odmocniny z dvoch pri mapách zavedených v súťaži je zanedbateľný.

Skúsme si teraz nadefinovať hernú mapu formálne.

\begin{define}
{\sl Herná mapa m*n} je ohodnotený graf v ohodnotením $w$ s m*n vrcholmi očíslovanými od $v_{1,1}$ až po $v_{m,n}$ 
s~jednoduchými hranami $j$ v~tvare $\{v_{a,b}, v_{a,b+1}\}, \{v_{a,b}, v_{a+1,b}\}$, kde $w(j) = 1$ 
a šikmými hranami $ s $ v~tvare $\{v_{a,b}, v_{a+1,b+1}\}, \{v_{a,b}, v_{a+1,b+1}\}$, kde $ w(s) = 1.4142 $.
\end{define}

FIXME?? pridat obrazok